\documentclass[a4paper,11pt]{article}
%\documentclass[preprint]{aa}

%\documentclass[preprint]{aastex}

%\documentclass[journal = ancham]{achemso}
%\setkeys{acs}{useutils = true}
%\usepackage{fullpage}
\usepackage{natbib,twoopt}
\pretolerance=2000
\tolerance=6000
\hbadness=6000
%\usepackage[landscape]{geometry}
%\usepackage{pxfonts}
%\usepackage{cmbright}
%\usepackage[varg]{txfonts}
%\usepackage{mathptmx}
%\usepackage{tgtermes}
\usepackage[utf8]{inputenc}
%\usepackage{fouriernc}
%\usepackage[adobe-utopia]{mathdesign}
\usepackage[T1]{fontenc}
%\usepackage[norsk]{babel}
\usepackage{epsfig}
\usepackage{graphicx}
\usepackage{amsmath}
%\usepackage[version=3]{mhchem}
\usepackage{pstricks}
\usepackage[font=small,labelfont=bf,tableposition=below]{caption}
%\usepackage{subfig}
\usepackage{subcaption}
%\usepackage{varioref}
\usepackage{hyperref}
\usepackage{listings}
%\usepackage{sverb}
\usepackage{verbatim}
%\usepackage{microtype}
%\usepackage{enumerate}
\usepackage{enumitem}
%\usepackage{lineno}
%\usepackage{booktabs}
%\usepackage{changepage}
%\usepackage[flushleft]{threeparttable}
\usepackage{pdfpages}
\usepackage{float}
\usepackage{mathtools}
%\usepackage{etoolbox}
%\usepackage{xstring}
\usepackage{aas_macros}
\usepackage[parfill]{parskip}

\floatstyle{plaintop}
\restylefloat{table}
%\floatsetup[table]{capposition=top}

\setcounter{secnumdepth}{3}

\newcommand{\tr}{\, \text{tr}\,}
\newcommand{\diff}{\ensuremath{\; \text{d}}}
\newcommand{\diffd}{\ensuremath{\text{d}}}
\newcommand{\sgn}{\ensuremath{\; \text{sgn}}}
\newcommand{\UA}{\ensuremath{_{\uparrow}}}
\newcommand{\RA}{\ensuremath{_{\rightarrow}}}
\newcommand{\QED}{\left\{ \hfill{\textbf{QED}} \right\}}
\newcommand{\CHHO}{\text{CH\ensuremath{_2}O} }

%% The below macros turn citations into ADS clickers in dvi, pdf, html output.
%% EDP Sciences improved them in December 2012 to work also with pdflatex.
\bibpunct{(}{)}{;}{a}{}{,}    %% natbib cite format used by A&A and ApJ
\makeatletter
 \newcommandtwoopt{\citeads}[3][][]{\href{http://adsabs.harvard.edu/abs/#3}%
   {\def\hyper@linkstart##1##2{}%
    \let\hyper@linkend\@empty\citealp[#1][#2]{#3}}}    %% Rutten, 2000
 \newcommandtwoopt{\citepads}[3][][]{\href{http://adsabs.harvard.edu/abs/#3}%
   {\def\hyper@linkstart##1##2{}%
    \let\hyper@linkend\@empty\citep[#1][#2]{#3}}}      %% (Rutten 2000)
 \newcommandtwoopt{\citetads}[3][][]{\href{http://adsabs.harvard.edu/abs/#3}%
   {\def\hyper@linkstart##1##2{}%
    \let\hyper@linkend\@empty\citet[#1][#2]{#3}}}      %% Rutten (2000)
 \newcommandtwoopt{\citeyearads}[3][][]%
   {\href{http://adsabs.harvard.edu/abs/#3}%
   {\def\hyper@linkstart##1##2{}%
    \let\hyper@linkend\@empty\citeyear[#1][#2]{#3}}}   %% 2000
\makeatother

%\newcommand{\diff}{%
%    \IfEqCase{frac{\diff}{%
%        {\ensuremath{frac{\text{d}} }}%
%        {\ensuremath{\; \text{d}} }% 
%    }[\PackageError{diff}{Problem with diff}{}]%
%}%


\date{\today}
\title{Project Exam spring 2014\\ \small{Statistical methods and applications -- STK4900}}
\author{Candidate} 


\begin{document}


\onecolumn
\maketitle{}


\section*{Problem 1}

\begin{enumerate}[label=\alph*)]
    \item The concentration in parts per million of formaldehyde in 24 houses is tested against an air tightness measure in the range $[0,10]$ and whether urea formaldehyde foam insulation (UFFI) has been used. Fig.~(\ref{fig:1a-1}) shows a scatter plot of the CH$_2$O concentration distributed against the air tightness, and fig.~(\ref{fig:1a-2}) shows a box plot where the CH$_2$O concentration distribution is shown as a function of UFFI presence. 

        The plots show that there both appears to be a positive correlation between increased air tightness and CH$_2$O concentration and also a positive correlation between UFFI presence and increased CH$_2$O concentration. However, the plots do not tell whether there is a relation between the presence of UFFI and the air tightness, and if so, these covariates on the response CH$_2$O concentration are \textit{confounded}. 

        Using univariate regression, where the response \CHHO is modelled as either a function of air tightness, or as a function of UFFI presence, described generally as
        \begin{equation}
            {\rm CH_2O} = \beta_0 = \beta_1 x_1 + \beta_2 x_2 + \dots + \varepsilon
            \label{eq:linmod}
        \end{equation}
        where $\beta_0$ gives the mean concentration of \CHHO (ie.,~\textit{the intercept}), and $\beta_1$ gives the change per unit change (ie.,~\textit{the slope}) in the covariate $x_1$ which either can be the air tightness, or the binary covariate UFFI presence in this part of the problem. The $\varepsilon$ is the error. 

        The coefficients of determination, giving a measure on how large the residual sum of squares are, compared to the total sum of squares, are given for the aforementioned models in tab.~(\ref{tab:1}). A smaller residual sum of squares, gives a larger $R^2$.  This is beneficial, because the variability is then minimised.

        Model 1 has $\beta_0 = 46.12$ and $\beta_1 = 9.074$ and the Student's $t$-value is 2.253 giving a $P$-value of 0.0346, which is large, meaning the null hypothesis $H_0$ (the covariate has no effect) is \textit{not} rejected.

        Model 2 has $\beta_0 = 36.12$ and $\beta_1 = 2.830$, with a $t$-value of 5.573 giving a $P$-value of $1.33\times10^{-5}$, which is small, meaning the null hypothesis $H_0$ is rejected.

        \begin{figure}[htb]
            \centering
            \includegraphics[width=0.6\columnwidth]{../1a1-CH2O_air.pdf}
            \caption{Scatter plot showing how the CH$_2$O concentration in parts per million (ppm) distribute according to the measured air tightness in Problem 1. By visual inspection, there appear to be a positive correlation, and Pearson's correlation coefficient is 0.7651. }
            \label{fig:1a-1}
        \end{figure}
        \begin{figure}[htb]
            \centering
            \includegraphics[width=0.6\columnwidth]{../1a2-CH2O_UFFI.pdf}
            \caption{Boxplot showing the response of the CH$_2$O-concentration depending on whether a home has UFFI present (category 1) or not (category 1). The median CH$_2$O concentration is higher when UFFI is present, than without.}
            \label{fig:1a-2}
        \end{figure} 
        \begin{figure}[htb]
            \centering
            \includegraphics[width=0.6\columnwidth]{../1a3-model_crossvalR2.pdf}
            \caption{The coefficients of determination $R^2$ (blue boxes), adjusted $R_2$ (black circles) and the cross validated $R^2_{\rm cv}$ (red filled circles) plotted against model number (given in tab.~(\ref{tab:1}). The highest cross validated coefficient of determination is achieved for model 3. This model is pursued further in problem 1c). }
            \label{fig:1a-3}
        \end{figure}
    \item Including a interaction term and both covariates, gives model 4. Both covariates alone is given by model 3. Model 5 includes, in addition to an interaction term, also a quadratic term for the air tightness. See tab.~(\ref{tab:1}). 

        According to the cross validated $R^2_{\rm cv}$, model 3 is the best model, with $\beta_0 = 31.34$, $\beta_{\rm AIR} = 2.855$ and $\beta_{\rm UFFI} = 9.312$. The $P$-values for both covariates are small (of magnitudes $10^{-7}$ and $10^{-4}$, respectively), thus are they likely to have significant effect on the outcome. 

        The interaction term \texttt{AIR:UFFI}, which would be represented in a linear model as the term $\beta_{\rm AIR:UFFI} x_{\rm AIR} x_{\rm UFFI}$, had a large $P$-value and is thus insignificant in combination with the other two covariates.

        Including also a quadratic term \texttt{I(AIR$^2$)}, given as $\beta_{\rm AIR^2} x_{\rm AIR}^2$, which neither is significant ($P = 0.5078$, accepting the null hypothesis), does not improve the model and reduces the significance of the other covariates.

        \begin{table}
            \scriptsize
            \centering
            \makebox[\linewidth]{
            \begin{tabular}{l c c c c c c c}
                Model   & \texttt{UFFI}     & \texttt{AIR} & \texttt{AIR:UFFI}  & \texttt{I(AIR$^2$)} & $R^2$ & Adj. $R^2$ & $R^2_{\rm cv}$ \\
                \hline
                1       & Y &&&& 0.1874 & 0.1505 & 0.1505 \\
                2       & & Y  &&& 0.5854 & 0.5665 & 0.5665  \\
                3       & Y & Y &&  & 0.7827 & 0.7620 &  0.7620 \\
                4       & Y & Y & Y & & 0.7895 & 0.7580 &0.7580 \\
                5       & Y & Y & Y & Y & 0.7899 & 0.7457 & 0.7457 \\
                \hline
            \end{tabular}}
            \caption{Models used in problem 1, showing what covariates that are included. The three rightmost columns give the $R^2$ coefficients, where the adjusted $R^2$ coefficient is less prone to outliers, and the cross validated $R^2_{\rm cv}$. The latter penalises the inclusion of additional covariates. See fig.~(\ref{fig:1a-3}) for a plot of the coefficients of determination.}
            \label{tab:1}
        \end{table}

    \item The residuals (the difference between the observed values, and the predicted values by the linear model) for model 3 are plotted in figs.~(\ref{fig:1c-1}, \ref{fig:1c-2}). See the caption text for details. The residuals do not appear to have any systematic pattern. 

        \begin{figure}[htb]
            \centering
            \includegraphics[width=0.6\columnwidth]{../1c1-model3_residuals.pdf}
            \caption{Plot showing the distribution using model 3, where the response CH$_2$O concentration in PPM is given by the two linear covariates \texttt{AIR} (density) and \texttt{UFFI} (presence of formaldehyde insulation foam). Note that the residuals do not appear to have any systematic pattern, they are thus distributed homoscedastically. The presence of outliers (denoted by numbers next to the points) make the model under- and overpredict the model response compared to the outlier's values. }
            \label{fig:1c-1}
        \end{figure}
        \begin{figure}[htb]
            \centering
            \includegraphics[width=0.6\columnwidth]{../1c2-model3_stdresiduals.pdf}
            \caption{Plot showing the distribution of \textit{standardised} residulas for model 3 with two linear covariates. The standardised residuals are less prone to outliers, in comparision with fig.~(\ref{fig:1c-1}), where the outliers affect the predicted outcome, there is no such pattern in the current plot. }
            \label{fig:1c-2}
        \end{figure}
\end{enumerate}

\section*{Problem 2}
\begin{enumerate}[label=\alph*)]

    \item a
\end{enumerate}


\begin{enumerate}[resume*]
    \item a
\end{enumerate}

%%%%%%%%%%% BIBLIOGRAPHY %%%%%%%%%%%%%%%%%
\bibliography{referanser}
\bibliographystyle{apj}
%\bibliographystyle{astroads}
%\bibliographystyle{apj_hyperref}

\clearpage
\appendix
\section{Appendix}
\label{sec:appendix}

\subsection{Problem 1a)}
\label{app:b}

For the linear models in exercise 1a);
{\footnotesize
    \verbatiminput{../1a_models.txt}
}

\subsection{Problem 1b)}
\label{app:b}

For the three linear models in exercise 1b);
{\footnotesize
    \verbatiminput{../1b_models.txt}
}


%\section{Source code}

%\lstinputlisting[language=R]{../bears.R}



\end{document}

